\documentclass[12pt]{article}

\usepackage{amsmath}
\usepackage{amsfonts}
\usepackage{graphicx}
\usepackage{hyperref}
\usepackage{pifont}

\title{Exploring Elliptic Curve Digital Signature Algorithm (ECDSA)}
\author{Thomas Harper}
\date{\today}

\begin{document}

\maketitle

\begin{abstract}

These are simply my notes on ECDSA, sufficient to complete the RareSkills ECDSA coursework.

\end{abstract}

\section{Introduction}

ECDSA is based upon the Elliptic Curve Discrete Logarithm Problem (ECDLP). The ECDLP is the problem of finding $k$ given $kP$ where $P$ is a point on an elliptic curve. The ECDLP is believed to be hard, and so ECDSA is believed to be secure.

\section{Elliptic Curves}

An elliptic curve has the following form:
$$y^2 = x^3 + ax + b$$
There are two constants, $a$ and $b$. The curve is defined over a finite field $\mathbb{F}_p$ where $p$ is a prime number. The curve is also defined over a point at infinity, denoted $\mathcal{O}$.

\section{secp256k1}

For secp256k1 specifically, a = 0 and b = 7:

$$y^2 = x^3 + 7$$

It is defined over a field $\mathbb{Z}_p$
\begin{itemize}
\item $\mathbb{Z}$ is the set of integers
\item $p$ is a prime number
    \begin{itemize}
        \item[\ding{223}] $p = 2^{256} - 2^{32} - 2^9 - 2^8 - 2^7 - 2^6 - 2^4 - 1$
        \item[\ding{223}] $p = 2^{256} - 2^{32} - 997$
        \item[\ding{223}] $p = FFFFFFFF FFFFFFFF FFFFFFFF FFFFFFFF FFFFFFFF FFFFFFFF FFFFFFFE FFFFFC2F$
    \end{itemize}
\item $\mathbb{Z}_p$ is the set of integers modulo $p$
\end{itemize}



% p = FFFFFFFF FFFFFFFF FFFFFFFF FFFFFFFF FFFFFFFF FFFFFFFF FFFFFFFE FFFFFC2F
% = 2256 - 232 - 29 - 28 - 27 - 26 - 24 - 1

\end{document}
